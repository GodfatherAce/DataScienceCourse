\documentclass[a4paper,12pt]{article}
%%%%%%%%%%%%%%%%%%%%%%%%%%%%%%%%%%%%%%%%%%%%%%%%%%%%%%%%%%%%%%%%%%%%%%%%%%%%%%%%%%%%%%%%%%%%%%%%%%%%%%%%%%%%%%%%%%%%%%%%%%%%%%%%%%%%%%%%%%%%%%%%%%%%%%%%%%%%%%%%%%%%%%%%%%%%%%%%%%%%%%%%%%%%%%%%%%%%%%%%%%%%%%%%%%%%%%%%%%%%%%%%%%%%%%%%%%%%%%%%%%%%%%%%%%%%
\usepackage{eurosym}
\usepackage{vmargin}
\usepackage{amsmath}
\usepackage{graphics}
\usepackage{epsfig}
\usepackage{subfigure}
\usepackage{fancyhdr}
%\usepackage{listings}
\usepackage{framed}
\usepackage{graphicx}
\usepackage{amsmath}
\usepackage{chngpage}
%\usepackage{bigints}
\setcounter{MaxMatrixCols}{10}
%TCIDATA{OutputFilter=LATEX.DLL}
%TCIDATA{Version=5.00.0.2570}
%TCIDATA{<META NAME="SaveForMode" CONTENT="1">}
%TCIDATA{LastRevised=Wednesday, February 23, 2011 13:24:34}
%TCIDATA{<META NAME="GraphicsSave" CONTENT="32">}
%TCIDATA{Language=American English}

\pagestyle{fancy}
\setmarginsrb{20mm}{0mm}{20mm}{25mm}{12mm}{11mm}{0mm}{11mm}
\lhead{Statistics} \rhead{Multivariate Statistics}
\chead{Logistic Regression}
%\input{tcilatex}

\begin{document}


\section*{Logistic function} 

The logistic function, with $\beta_0 + \beta_1 x$ on the horizontal axis and $\pi(x)$ on the vertical axis
An explanation of logistic regression begins with an explanation of the logistic function, which always takes on values between zero and one:
\[
F(t) = \frac{e^t}{e^t+1} = \frac{1}{1+e^{-t}},
\]
and viewing t as a linear function of an explanatory variable x (or of a linear combination of explanatory variables), the logistic function can be written as:
\[\pi(x) = \frac{e^{(\beta_0 + \beta_1 x)}} {e^{(\beta_0 + \beta_1 x)} + 1} = \frac {1} {1+e^{-(\beta_0 + \beta_1 x)}}.
\]
This will be interpreted as the probability of the dependent variable equalling a "success" or "case" rather than a failure or non-case. We also define the inverse of the logistic function, the logit:
\[g(x) = \ln \frac {\pi(x)} {1 - \pi(x)} = \beta_0 + \beta_1 x ,
\]and equivalently:
\[\frac{\pi(x)} {1 - \pi(x)} = e^{(\beta_0 + \beta_1 x)}.
\]

\section*{Odds Ratio - Example}

Suppose that in a sample of 100 men, 90 drank wine in the previous week, while in a sample of 100 women only 20 drank wine in the same period. The odds of a man drinking wine are 90 to 10, or 9:1, while the odds of a woman drinking wine are only 20 to 80, or 1:4 = 0.25:1. The odds ratio is thus 9/0.25, or 36, showing that men are much more likely to drink wine than women. The detailed calculation is:
\[{ 0.9/0.1 \over 0.2/0.8}=\frac{\;0.9\times 0.8\;}{\;0.1\times 0.2\;} ={0.72 \over 0.02} = 36.\]
This example also shows how odds ratios are sometimes sensitive in stating relative positions: in this sample men are 90/20 = 4.5 times more likely to have drunk wine than women, but have 36 times the odds. The logarithm of the odds ratio, the difference of the logits of the probabilities, tempers this effect, and also makes the measure symmetric with respect to the ordering of groups. For example, using natural logarithms, an odds ratio of 36/1 maps to 3.584, and an odds ratio of 1/36 maps to −3.584.
\section*{Logit}


\end{document}

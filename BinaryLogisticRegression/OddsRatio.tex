\section*{Odds Ratio - Example}

Suppose that in a sample of 100 men, 90 drank wine in the previous week, while in a sample of 100 women only 20 drank wine in the same period. The odds of a man drinking wine are 90 to 10, or 9:1, while the odds of a woman drinking wine are only 20 to 80, or 1:4 = 0.25:1. The odds ratio is thus 9/0.25, or 36, showing that men are much more likely to drink wine than women. The detailed calculation is:
\[{ 0.9/0.1 \over 0.2/0.8}=\frac{\;0.9\times 0.8\;}{\;0.1\times 0.2\;} ={0.72 \over 0.02} = 36.\]
This example also shows how odds ratios are sometimes sensitive in stating relative positions: in this sample men are 90/20 = 4.5 times more likely to have drunk wine than women, but have 36 times the odds. The logarithm of the odds ratio, the difference of the logits of the probabilities, tempers this effect, and also makes the measure symmetric with respect to the ordering of groups. For example, using natural logarithms, an odds ratio of 36/1 maps to 3.584, and an odds ratio of 1/36 maps to −3.584.

\section*{What is an odds ratio?}


An odds ratio (OR) is a measure of association between an exposure and an outcome. The OR represents the odds that an outcome will occur given a particular exposure, compared to the odds of the outcome occurring in the absence of that exposure. Odds ratios are most commonly used in case-control studies, however they can also be used in cross-sectional and cohort study designs as well (with some modifications and/or assumptions).

\subsection*{Odds ratios and logistic regression}
When a logistic regression is calculated, the regression coefficient (b1) is the estimated increase in the log odds of the outcome per unit increase in the value of the exposure. In other words, the exponential function of the regression coefficient (eb1) is the odds ratio associated with a one-unit increase in the exposure.


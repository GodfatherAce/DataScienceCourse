\documentclass[a4paper,12pt]{article}
%%%%%%%%%%%%%%%%%%%%%%%%%%%%%%%%%%%%%%%%%%%%%%%%%%%%%%%%%%%%%%%%%%%%%%%%%%%%%%%%%%%%%%%%%%%%%%%%%%%%%%%%%%%%%%%%%%%%%%%%%%%%%%%%%%%%%%%%%%%%%%%%%%%%%%%%%%%%%%%%%%%%%%%%%%%%%%%%%%%%%%%%%%%%%%%%%%%%%%%%%%%%%%%%%%%%%%%%%%%%%%%%%%%%%%%%%%%%%%%%%%%%%%%%%%%%
\usepackage{eurosym}
\usepackage{vmargin}
\usepackage{amsmath}
\usepackage{graphics}
\usepackage{epsfig}
\usepackage{subfigure}
\usepackage{fancyhdr}
%\usepackage{listings}
\usepackage{framed}
\usepackage{graphicx}
\usepackage{amsmath}
\usepackage{chngpage}
%\usepackage{bigints}
\setcounter{MaxMatrixCols}{10}
%TCIDATA{OutputFilter=LATEX.DLL}
%TCIDATA{Version=5.00.0.2570}
%TCIDATA{<META NAME="SaveForMode" CONTENT="1">}
%TCIDATA{LastRevised=Wednesday, February 23, 2011 13:24:34}
%TCIDATA{<META NAME="GraphicsSave" CONTENT="32">}
%TCIDATA{Language=American English}

\pagestyle{fancy}
\setmarginsrb{20mm}{0mm}{20mm}{25mm}{12mm}{11mm}{0mm}{11mm}
\lhead{Statistics} \rhead{Multivariate Statistics}
\chead{Logistic Regression}
%\input{tcilatex}

\begin{document}


\section*{Logistic function} 

The logistic function, with $\beta_0 + \beta_1 x$ on the horizontal axis and $\pi(x)$ on the vertical axis
An explanation of logistic regression begins with an explanation of the logistic function, which always takes on values between zero and one:
\[
F(t) = \frac{e^t}{e^t+1} = \frac{1}{1+e^{-t}},
\]
and viewing t as a linear function of an explanatory variable x (or of a linear combination of explanatory variables), the logistic function can be written as:
\[\pi(x) = \frac{e^{(\beta_0 + \beta_1 x)}} {e^{(\beta_0 + \beta_1 x)} + 1} = \frac {1} {1+e^{-(\beta_0 + \beta_1 x)}}.
\]
This will be interpreted as the probability of the dependent variable equalling a "success" or "case" rather than a failure or non-case. We also define the inverse of the logistic function, the logit:
\[g(x) = \ln \frac {\pi(x)} {1 - \pi(x)} = \beta_0 + \beta_1 x ,
\]and equivalently:
\[\frac{\pi(x)} {1 - \pi(x)} = e^{(\beta_0 + \beta_1 x)}.
\]



\end{document}
